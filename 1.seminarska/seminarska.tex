% Options for packages loaded elsewhere
\PassOptionsToPackage{unicode}{hyperref}
\PassOptionsToPackage{hyphens}{url}
%
\documentclass[
]{article}
\title{Umetna inteligenca 2021-2022}
\usepackage{etoolbox}
\makeatletter
\providecommand{\subtitle}[1]{% add subtitle to \maketitle
  \apptocmd{\@title}{\par {\large #1 \par}}{}{}
}
\makeatother
\subtitle{Seminarska naloga 1}
\author{Aljaž Hribar}
\date{30 November 2021}

\usepackage{amsmath,amssymb}
\usepackage{lmodern}
\usepackage{iftex}
\ifPDFTeX
  \usepackage[T1]{fontenc}
  \usepackage[utf8]{inputenc}
  \usepackage{textcomp} % provide euro and other symbols
\else % if luatex or xetex
  \usepackage{unicode-math}
  \defaultfontfeatures{Scale=MatchLowercase}
  \defaultfontfeatures[\rmfamily]{Ligatures=TeX,Scale=1}
\fi
% Use upquote if available, for straight quotes in verbatim environments
\IfFileExists{upquote.sty}{\usepackage{upquote}}{}
\IfFileExists{microtype.sty}{% use microtype if available
  \usepackage[]{microtype}
  \UseMicrotypeSet[protrusion]{basicmath} % disable protrusion for tt fonts
}{}
\makeatletter
\@ifundefined{KOMAClassName}{% if non-KOMA class
  \IfFileExists{parskip.sty}{%
    \usepackage{parskip}
  }{% else
    \setlength{\parindent}{0pt}
    \setlength{\parskip}{6pt plus 2pt minus 1pt}}
}{% if KOMA class
  \KOMAoptions{parskip=half}}
\makeatother
\usepackage{xcolor}
\IfFileExists{xurl.sty}{\usepackage{xurl}}{} % add URL line breaks if available
\IfFileExists{bookmark.sty}{\usepackage{bookmark}}{\usepackage{hyperref}}
\hypersetup{
  pdftitle={Umetna inteligenca 2021-2022},
  pdfauthor={Aljaž Hribar},
  hidelinks,
  pdfcreator={LaTeX via pandoc}}
\urlstyle{same} % disable monospaced font for URLs
\usepackage[margin=1in]{geometry}
\usepackage{color}
\usepackage{fancyvrb}
\newcommand{\VerbBar}{|}
\newcommand{\VERB}{\Verb[commandchars=\\\{\}]}
\DefineVerbatimEnvironment{Highlighting}{Verbatim}{commandchars=\\\{\}}
% Add ',fontsize=\small' for more characters per line
\usepackage{framed}
\definecolor{shadecolor}{RGB}{248,248,248}
\newenvironment{Shaded}{\begin{snugshade}}{\end{snugshade}}
\newcommand{\AlertTok}[1]{\textcolor[rgb]{0.94,0.16,0.16}{#1}}
\newcommand{\AnnotationTok}[1]{\textcolor[rgb]{0.56,0.35,0.01}{\textbf{\textit{#1}}}}
\newcommand{\AttributeTok}[1]{\textcolor[rgb]{0.77,0.63,0.00}{#1}}
\newcommand{\BaseNTok}[1]{\textcolor[rgb]{0.00,0.00,0.81}{#1}}
\newcommand{\BuiltInTok}[1]{#1}
\newcommand{\CharTok}[1]{\textcolor[rgb]{0.31,0.60,0.02}{#1}}
\newcommand{\CommentTok}[1]{\textcolor[rgb]{0.56,0.35,0.01}{\textit{#1}}}
\newcommand{\CommentVarTok}[1]{\textcolor[rgb]{0.56,0.35,0.01}{\textbf{\textit{#1}}}}
\newcommand{\ConstantTok}[1]{\textcolor[rgb]{0.00,0.00,0.00}{#1}}
\newcommand{\ControlFlowTok}[1]{\textcolor[rgb]{0.13,0.29,0.53}{\textbf{#1}}}
\newcommand{\DataTypeTok}[1]{\textcolor[rgb]{0.13,0.29,0.53}{#1}}
\newcommand{\DecValTok}[1]{\textcolor[rgb]{0.00,0.00,0.81}{#1}}
\newcommand{\DocumentationTok}[1]{\textcolor[rgb]{0.56,0.35,0.01}{\textbf{\textit{#1}}}}
\newcommand{\ErrorTok}[1]{\textcolor[rgb]{0.64,0.00,0.00}{\textbf{#1}}}
\newcommand{\ExtensionTok}[1]{#1}
\newcommand{\FloatTok}[1]{\textcolor[rgb]{0.00,0.00,0.81}{#1}}
\newcommand{\FunctionTok}[1]{\textcolor[rgb]{0.00,0.00,0.00}{#1}}
\newcommand{\ImportTok}[1]{#1}
\newcommand{\InformationTok}[1]{\textcolor[rgb]{0.56,0.35,0.01}{\textbf{\textit{#1}}}}
\newcommand{\KeywordTok}[1]{\textcolor[rgb]{0.13,0.29,0.53}{\textbf{#1}}}
\newcommand{\NormalTok}[1]{#1}
\newcommand{\OperatorTok}[1]{\textcolor[rgb]{0.81,0.36,0.00}{\textbf{#1}}}
\newcommand{\OtherTok}[1]{\textcolor[rgb]{0.56,0.35,0.01}{#1}}
\newcommand{\PreprocessorTok}[1]{\textcolor[rgb]{0.56,0.35,0.01}{\textit{#1}}}
\newcommand{\RegionMarkerTok}[1]{#1}
\newcommand{\SpecialCharTok}[1]{\textcolor[rgb]{0.00,0.00,0.00}{#1}}
\newcommand{\SpecialStringTok}[1]{\textcolor[rgb]{0.31,0.60,0.02}{#1}}
\newcommand{\StringTok}[1]{\textcolor[rgb]{0.31,0.60,0.02}{#1}}
\newcommand{\VariableTok}[1]{\textcolor[rgb]{0.00,0.00,0.00}{#1}}
\newcommand{\VerbatimStringTok}[1]{\textcolor[rgb]{0.31,0.60,0.02}{#1}}
\newcommand{\WarningTok}[1]{\textcolor[rgb]{0.56,0.35,0.01}{\textbf{\textit{#1}}}}
\usepackage{graphicx}
\makeatletter
\def\maxwidth{\ifdim\Gin@nat@width>\linewidth\linewidth\else\Gin@nat@width\fi}
\def\maxheight{\ifdim\Gin@nat@height>\textheight\textheight\else\Gin@nat@height\fi}
\makeatother
% Scale images if necessary, so that they will not overflow the page
% margins by default, and it is still possible to overwrite the defaults
% using explicit options in \includegraphics[width, height, ...]{}
\setkeys{Gin}{width=\maxwidth,height=\maxheight,keepaspectratio}
% Set default figure placement to htbp
\makeatletter
\def\fps@figure{htbp}
\makeatother
\setlength{\emergencystretch}{3em} % prevent overfull lines
\providecommand{\tightlist}{%
  \setlength{\itemsep}{0pt}\setlength{\parskip}{0pt}}
\setcounter{secnumdepth}{-\maxdimen} % remove section numbering
\ifLuaTeX
  \usepackage{selnolig}  % disable illegal ligatures
\fi

\begin{document}
\maketitle

\hypertarget{klasifikacijski-problem}{%
\section{Klasifikacijski problem}\label{klasifikacijski-problem}}

\hypertarget{ocenjevanje-in-konstrukcija-atributov}{%
\subsection{Ocenjevanje in konstrukcija
atributov}\label{ocenjevanje-in-konstrukcija-atributov}}

najprej faktoriziramo vse atribute, ki niso zvezni

\begin{Shaded}
\begin{Highlighting}[]
\NormalTok{ucna}\SpecialCharTok{$}\NormalTok{regija}\OtherTok{\textless{}{-}}\FunctionTok{as.factor}\NormalTok{(ucna}\SpecialCharTok{$}\NormalTok{regija)}
\NormalTok{testna}\SpecialCharTok{$}\NormalTok{regija}\OtherTok{\textless{}{-}}\FunctionTok{as.factor}\NormalTok{(testna}\SpecialCharTok{$}\NormalTok{regija)}
\NormalTok{ucna}\SpecialCharTok{$}\NormalTok{namembnost}\OtherTok{\textless{}{-}}\FunctionTok{as.factor}\NormalTok{(ucna}\SpecialCharTok{$}\NormalTok{namembnost)}
\NormalTok{testna}\SpecialCharTok{$}\NormalTok{namembnost}\OtherTok{\textless{}{-}}\FunctionTok{as.factor}\NormalTok{(testna}\SpecialCharTok{$}\NormalTok{namembnost)}
\NormalTok{ucna}\SpecialCharTok{$}\NormalTok{oblacnost}\OtherTok{\textless{}{-}}\FunctionTok{as.factor}\NormalTok{(ucna}\SpecialCharTok{$}\NormalTok{oblacnost)}
\NormalTok{testna}\SpecialCharTok{$}\NormalTok{oblacnost}\OtherTok{\textless{}{-}}\FunctionTok{as.factor}\NormalTok{(testna}\SpecialCharTok{$}\NormalTok{oblacnost)}
\FunctionTok{summary}\NormalTok{(ucna)}
\end{Highlighting}
\end{Shaded}

\begin{verbatim}
##     datum               regija          stavba      
##  Length:24125       vzhodna:11315   Min.   :  1.00  
##  Class :character   zahodna:12810   1st Qu.: 39.00  
##  Mode  :character                   Median : 79.00  
##                                     Mean   : 87.49  
##                                     3rd Qu.:135.00  
##                                     Max.   :193.00  
##                 namembnost       povrsina       leto_izgradnje   temp_zraka   
##  izobrazevalna       :13301   Min.   :  329.3   Min.   :1903   Min.   :-7.20  
##  javno_storitvena    : 2979   1st Qu.: 4106.6   1st Qu.:1950   1st Qu.:10.00  
##  kulturno_razvedrilna: 3263   Median : 6763.3   Median :1970   Median :20.00  
##  poslovna            : 3057   Mean   :10958.1   Mean   :1970   Mean   :19.15  
##  stanovanjska        : 1525   3rd Qu.:14409.3   3rd Qu.:2000   3rd Qu.:28.30  
##                               Max.   :79000.4   Max.   :2017   Max.   :41.70  
##   temp_rosisca     oblacnost    padavine          pritisk         smer_vetra   
##  Min.   :-19.400   0:3090    Min.   :-1.0000   Min.   : 997.2   Min.   :  0.0  
##  1st Qu.: -2.800   2:8390    1st Qu.: 0.0000   1st Qu.:1011.9   1st Qu.: 70.0  
##  Median :  2.800   4:4514    Median : 0.0000   Median :1015.9   Median :140.0  
##  Mean   :  3.816   6:5126    Mean   : 0.3113   Mean   :1017.1   Mean   :156.6  
##  3rd Qu.: 11.100   8:2950    3rd Qu.: 0.0000   3rd Qu.:1021.8   3rd Qu.:250.0  
##  Max.   : 25.000   9:  55    Max.   :56.0000   Max.   :1040.9   Max.   :360.0  
##  hitrost_vetra        poraba       
##  Min.   : 0.000   Min.   :   0.00  
##  1st Qu.: 2.100   1st Qu.:  53.48  
##  Median : 3.600   Median : 112.90  
##  Mean   : 3.756   Mean   : 224.55  
##  3rd Qu.: 5.100   3rd Qu.: 215.41  
##  Max.   :12.400   Max.   :2756.54
\end{verbatim}

opazimo da lahko je smer vetra podana koz zvezni podatek ampak bi nam
bila bolj uporabna kot diskretni zato jo faktoriziramo

\begin{Shaded}
\begin{Highlighting}[]
\FunctionTok{table}\NormalTok{(ucna}\SpecialCharTok{$}\NormalTok{smer\_vetra)}
\end{Highlighting}
\end{Shaded}

\begin{verbatim}
## 
##  brezveterje          jug   jugo_vzhod   jugo_zahod        sever severo_vzhod 
##         2482         4210         3325         1044         1740         2670 
## severo_zahod        vzhod        zahod 
##         2831         3799         2024
\end{verbatim}

iz atributa ``datum'' lahko generiramo nov atribut ``season'', ki nam
pove letni čas meritve in atribut ``vikend'' ki nam pove ali je na ta
datum bil vikend ali delovni teden

\begin{Shaded}
\begin{Highlighting}[]
\FunctionTok{table}\NormalTok{(ucna}\SpecialCharTok{$}\NormalTok{season)}
\end{Highlighting}
\end{Shaded}

\begin{verbatim}
## 
##   Fall Spring Summer Winter 
##   6741   4235   5140   8009
\end{verbatim}

\begin{Shaded}
\begin{Highlighting}[]
\FunctionTok{table}\NormalTok{(ucna}\SpecialCharTok{$}\NormalTok{vikend)}
\end{Highlighting}
\end{Shaded}

\begin{verbatim}
## 
## FALSE  TRUE 
## 17184  6941
\end{verbatim}

prav tako lahko iz atributa ``poraba'' izvlečemo atributa
``dosedanja\_povpreča'' in ``dosedanja\_skupna'' ki nam povesta kolikšna
je povprečna in skupna poraba stavbe do vključno trenutnega datuma
meritve

\begin{Shaded}
\begin{Highlighting}[]
\FunctionTok{summary}\NormalTok{(ucna}\SpecialCharTok{$}\NormalTok{dosedanja\_povprecna)}
\end{Highlighting}
\end{Shaded}

\begin{verbatim}
##     Min.  1st Qu.   Median     Mean  3rd Qu.     Max. 
##    3.616   60.733  118.774  228.265  204.453 2196.688
\end{verbatim}

\begin{Shaded}
\begin{Highlighting}[]
\FunctionTok{summary}\NormalTok{(ucna}\SpecialCharTok{$}\NormalTok{dosedanja\_skupna)}
\end{Highlighting}
\end{Shaded}

\begin{verbatim}
##     Min.  1st Qu.   Median     Mean  3rd Qu.     Max. 
##      3.7   4531.1  12638.2  28874.0  30728.0 424149.1
\end{verbatim}

sedaj lahko izločimo atribut stavba saj je za klasifikacijo odvečen
atribut, ki bi samo kvaril modele

\begin{Shaded}
\begin{Highlighting}[]
\NormalTok{ucna}\SpecialCharTok{$}\NormalTok{stavba}\OtherTok{\textless{}{-}}\ConstantTok{NULL}
\NormalTok{testna}\SpecialCharTok{$}\NormalTok{stavba}\OtherTok{\textless{}{-}}\ConstantTok{NULL}
\end{Highlighting}
\end{Shaded}

z attrEval() funkcijo ocenimo atribute,

\begin{Shaded}
\begin{Highlighting}[]
\FunctionTok{library}\NormalTok{(CORElearn)}
\end{Highlighting}
\end{Shaded}

\begin{verbatim}
## Warning: package 'CORElearn' was built under R version 4.0.5
\end{verbatim}

\begin{Shaded}
\begin{Highlighting}[]
\FunctionTok{sort}\NormalTok{(}\FunctionTok{attrEval}\NormalTok{(namembnost }\SpecialCharTok{\textasciitilde{}}\NormalTok{ ., ucna, }\StringTok{"Relief"}\NormalTok{), }\AttributeTok{decreasing =} \ConstantTok{TRUE}\NormalTok{)}
\end{Highlighting}
\end{Shaded}

\begin{verbatim}
## dosedanja_povprecna            povrsina              poraba    dosedanja_skupna 
##         0.061888110         0.035969465         0.035938575         0.035583211 
##      leto_izgradnje              regija            padavine              season 
##         0.016055510         0.000000000        -0.000613520        -0.002901554 
##              vikend          temp_zraka        temp_rosisca             pritisk 
##        -0.005015544        -0.007857875        -0.011347804        -0.013097497 
##           oblacnost          smer_vetra       hitrost_vetra               datum 
##        -0.013471503        -0.015046632        -0.015718982        -0.057616580
\end{verbatim}

\begin{Shaded}
\begin{Highlighting}[]
\FunctionTok{sort}\NormalTok{(}\FunctionTok{attrEval}\NormalTok{(namembnost }\SpecialCharTok{\textasciitilde{}}\NormalTok{ ., ucna, }\StringTok{"ReliefFequalK"}\NormalTok{), }\AttributeTok{decreasing =} \ConstantTok{TRUE}\NormalTok{)}
\end{Highlighting}
\end{Shaded}

\begin{verbatim}
##      leto_izgradnje            povrsina              regija dosedanja_povprecna 
##         0.242451468         0.166456481         0.139201471         0.136221147 
##              poraba    dosedanja_skupna               datum          temp_zraka 
##         0.115882015         0.105300619         0.102695830         0.082560008 
##          smer_vetra             pritisk        temp_rosisca       hitrost_vetra 
##         0.067548928         0.062092321         0.047547866         0.040365334 
##           oblacnost              season              vikend            padavine 
##         0.038386700         0.027151592         0.007127530         0.004231328
\end{verbatim}

\begin{Shaded}
\begin{Highlighting}[]
\FunctionTok{sort}\NormalTok{(}\FunctionTok{attrEval}\NormalTok{(namembnost }\SpecialCharTok{\textasciitilde{}}\NormalTok{ ., ucna, }\StringTok{"ReliefFexpRank"}\NormalTok{), }\AttributeTok{decreasing =} \ConstantTok{TRUE}\NormalTok{)}
\end{Highlighting}
\end{Shaded}

\begin{verbatim}
##      leto_izgradnje            povrsina              regija               datum 
##         0.260707133         0.166830040         0.164866503         0.119054407 
## dosedanja_povprecna              poraba          temp_zraka    dosedanja_skupna 
##         0.115823728         0.095175641         0.094856759         0.085446344 
##          smer_vetra             pritisk        temp_rosisca           oblacnost 
##         0.079984195         0.074110398         0.056642911         0.048480491 
##       hitrost_vetra              season              vikend            padavine 
##         0.046881204         0.033750883         0.008384743         0.004979457
\end{verbatim}

iz avaluacije atributov opazimo da imajo atributi ki opisujejo vremenske
razmere in datum meritev le teh zelo majheno povezavo z namembnostjo
stavbe zato jih lahko izločimo

\begin{Shaded}
\begin{Highlighting}[]
\NormalTok{ucna}\SpecialCharTok{$}\NormalTok{temp\_zraka}\OtherTok{\textless{}{-}}\ConstantTok{NULL}
\NormalTok{ucna}\SpecialCharTok{$}\NormalTok{pritisk}\OtherTok{\textless{}{-}}\ConstantTok{NULL}
\NormalTok{ucna}\SpecialCharTok{$}\NormalTok{temp\_rosisca}\OtherTok{\textless{}{-}}\ConstantTok{NULL}
\NormalTok{ucna}\SpecialCharTok{$}\NormalTok{padavine}\OtherTok{\textless{}{-}}\ConstantTok{NULL}
\NormalTok{ucna}\SpecialCharTok{$}\NormalTok{hitrost\_vetra}\OtherTok{\textless{}{-}}\ConstantTok{NULL}
\NormalTok{ucna}\SpecialCharTok{$}\NormalTok{smer\_vetra}\OtherTok{\textless{}{-}}\ConstantTok{NULL}
\NormalTok{ucna}\SpecialCharTok{$}\NormalTok{oblacnost}\OtherTok{\textless{}{-}}\ConstantTok{NULL}
\NormalTok{ucna}\SpecialCharTok{$}\NormalTok{datum}\OtherTok{\textless{}{-}}\ConstantTok{NULL}
\NormalTok{ucna}\SpecialCharTok{$}\NormalTok{season}\OtherTok{\textless{}{-}}\ConstantTok{NULL}
\NormalTok{ucna}\SpecialCharTok{$}\NormalTok{vikend}\OtherTok{\textless{}{-}}\ConstantTok{NULL}
\NormalTok{testna}\SpecialCharTok{$}\NormalTok{temp\_zraka}\OtherTok{\textless{}{-}}\ConstantTok{NULL}
\NormalTok{testna}\SpecialCharTok{$}\NormalTok{pritisk}\OtherTok{\textless{}{-}}\ConstantTok{NULL}
\NormalTok{testna}\SpecialCharTok{$}\NormalTok{temp\_rosisca}\OtherTok{\textless{}{-}}\ConstantTok{NULL}
\NormalTok{testna}\SpecialCharTok{$}\NormalTok{padavine}\OtherTok{\textless{}{-}}\ConstantTok{NULL}
\NormalTok{testna}\SpecialCharTok{$}\NormalTok{hitrost\_vetra}\OtherTok{\textless{}{-}}\ConstantTok{NULL}
\NormalTok{testna}\SpecialCharTok{$}\NormalTok{smer\_vetra}\OtherTok{\textless{}{-}}\ConstantTok{NULL}
\NormalTok{testna}\SpecialCharTok{$}\NormalTok{oblacnost}\OtherTok{\textless{}{-}}\ConstantTok{NULL}
\NormalTok{testna}\SpecialCharTok{$}\NormalTok{datum}\OtherTok{\textless{}{-}}\ConstantTok{NULL}
\NormalTok{testna}\SpecialCharTok{$}\NormalTok{season}\OtherTok{\textless{}{-}}\ConstantTok{NULL}
\NormalTok{testna}\SpecialCharTok{$}\NormalTok{vikend}\OtherTok{\textless{}{-}}\ConstantTok{NULL}
\NormalTok{ucna}\SpecialCharTok{$}\NormalTok{dosedanja\_skupna}\OtherTok{\textless{}{-}}\ConstantTok{NULL}
\NormalTok{testna}\SpecialCharTok{$}\NormalTok{dosedanja\_skupna}\OtherTok{\textless{}{-}}\ConstantTok{NULL}
\end{Highlighting}
\end{Shaded}

\hypertarget{gradnja-modelov}{%
\subsection{gradnja modelov}\label{gradnja-modelov}}

\hypertarget{odloux10ditveno-drevo}{%
\subsubsection{odločitveno drevo}\label{odloux10ditveno-drevo}}

najprej sva zgradila model z vsemi atributi

\begin{Shaded}
\begin{Highlighting}[]
\FunctionTok{rpart.plot}\NormalTok{(dt)}
\end{Highlighting}
\end{Shaded}

\begin{verbatim}
## Warning: labs do not fit even at cex 0.15, there may be some overplotting
\end{verbatim}

\includegraphics{seminarska_files/figure-latex/unnamed-chunk-16-1.pdf}

\begin{Shaded}
\begin{Highlighting}[]
\FunctionTok{CA}\NormalTok{(observed,predicted)}
\end{Highlighting}
\end{Shaded}

\begin{verbatim}
## [1] 0.5093227
\end{verbatim}

\begin{Shaded}
\begin{Highlighting}[]
\NormalTok{predMat }\OtherTok{\textless{}{-}} \FunctionTok{predict}\NormalTok{(dt, testna, }\AttributeTok{type =} \StringTok{"prob"}\NormalTok{)}
\FunctionTok{brier.score}\NormalTok{(obsMat, predMat)}
\end{Highlighting}
\end{Shaded}

\begin{verbatim}
## [1] 0.9812315
\end{verbatim}

nato pa še poiskusila minimalizirati napako z fukcijo wrapper(), ki je
vrnila da so najboljšo točnost imeli atributi povrsina in
leto\_izgradnje z pričakovano napako 0.007543904

\begin{Shaded}
\begin{Highlighting}[]
\FunctionTok{rpart.plot}\NormalTok{(dt)}
\end{Highlighting}
\end{Shaded}

\begin{verbatim}
## Warning: labs do not fit even at cex 0.15, there may be some overplotting
\end{verbatim}

\includegraphics{seminarska_files/figure-latex/unnamed-chunk-18-1.pdf}

\begin{Shaded}
\begin{Highlighting}[]
\FunctionTok{CA}\NormalTok{(observed,predicted)}
\end{Highlighting}
\end{Shaded}

\begin{verbatim}
## [1] 0.5005853
\end{verbatim}

\begin{Shaded}
\begin{Highlighting}[]
\NormalTok{predMat }\OtherTok{\textless{}{-}} \FunctionTok{predict}\NormalTok{(dt, testna, }\AttributeTok{type =} \StringTok{"prob"}\NormalTok{)}
\FunctionTok{brier.score}\NormalTok{(obsMat, predMat)}
\end{Highlighting}
\end{Shaded}

\begin{verbatim}
## [1] 0.9812315
\end{verbatim}

ko sva pognal wraper() za minimizacijo ``brier score'' sva dobila
podobne rezultate saj nama je funcija vrnila ``best model: estimated
error = 0.01390878 , selected feature subset = namembnost
\textasciitilde{} povrsina + leto\_izgradnje''

\begin{Shaded}
\begin{Highlighting}[]
\FunctionTok{rpart.plot}\NormalTok{(dt)}
\end{Highlighting}
\end{Shaded}

\begin{verbatim}
## Warning: labs do not fit even at cex 0.15, there may be some overplotting
\end{verbatim}

\includegraphics{seminarska_files/figure-latex/unnamed-chunk-20-1.pdf}

\begin{Shaded}
\begin{Highlighting}[]
\FunctionTok{CA}\NormalTok{(observed,predicted)}
\end{Highlighting}
\end{Shaded}

\begin{verbatim}
## [1] 0.5005853
\end{verbatim}

\begin{Shaded}
\begin{Highlighting}[]
\NormalTok{predMat }\OtherTok{\textless{}{-}} \FunctionTok{predict}\NormalTok{(dt, testna, }\AttributeTok{type =} \StringTok{"prob"}\NormalTok{)}
\FunctionTok{brier.score}\NormalTok{(obsMat, predMat)}
\end{Highlighting}
\end{Shaded}

\begin{verbatim}
## [1] 0.9988294
\end{verbatim}

\hypertarget{naivni-bayes}{%
\subsubsection{naivni bayes}\label{naivni-bayes}}

tukaj sva uporabila enak postopek kot pri gradnji drevesa in dobila
naslednje rezultate:

z vsemi atributi:

\begin{Shaded}
\begin{Highlighting}[]
\FunctionTok{CA}\NormalTok{(observed, predicted)}
\end{Highlighting}
\end{Shaded}

\begin{verbatim}
## [1] 0.4540552
\end{verbatim}

\begin{Shaded}
\begin{Highlighting}[]
\NormalTok{predMat }\OtherTok{\textless{}{-}} \FunctionTok{predict}\NormalTok{(nb, testna, }\AttributeTok{type =} \StringTok{"prob"}\NormalTok{)}
\FunctionTok{brier.score}\NormalTok{(obsMat, predMat)}
\end{Highlighting}
\end{Shaded}

\begin{verbatim}
## [1] 0.7545656
\end{verbatim}

wrapper z minimizacijo napake:

\begin{Shaded}
\begin{Highlighting}[]
\FunctionTok{CA}\NormalTok{(observed, predicted)}
\end{Highlighting}
\end{Shaded}

\begin{verbatim}
## [1] 0.4909699
\end{verbatim}

\begin{Shaded}
\begin{Highlighting}[]
\NormalTok{predMat }\OtherTok{\textless{}{-}} \FunctionTok{predict}\NormalTok{(nb, testna, }\AttributeTok{type =} \StringTok{"prob"}\NormalTok{)}
\FunctionTok{brier.score}\NormalTok{(obsMat, predMat)}
\end{Highlighting}
\end{Shaded}

\begin{verbatim}
## [1] 0.7259014
\end{verbatim}

wrapper z minimizacijo brier:

\begin{Shaded}
\begin{Highlighting}[]
\FunctionTok{CA}\NormalTok{(observed, predicted)}
\end{Highlighting}
\end{Shaded}

\begin{verbatim}
## [1] 0.4909699
\end{verbatim}

\begin{Shaded}
\begin{Highlighting}[]
\NormalTok{predMat }\OtherTok{\textless{}{-}} \FunctionTok{predict}\NormalTok{(nb, testna, }\AttributeTok{type =} \StringTok{"prob"}\NormalTok{)}
\FunctionTok{brier.score}\NormalTok{(obsMat, predMat)}
\end{Highlighting}
\end{Shaded}

\begin{verbatim}
## [1] 0.7259014
\end{verbatim}

\hypertarget{k-najbliux17ejih-sosedov}{%
\subsubsection{k-najbližjih sosedov}\label{k-najbliux17ejih-sosedov}}

\begin{Shaded}
\begin{Highlighting}[]
\FunctionTok{CA}\NormalTok{(observed, predicted)}
\end{Highlighting}
\end{Shaded}

\begin{verbatim}
## [1] 0.4678512
\end{verbatim}

\begin{Shaded}
\begin{Highlighting}[]
\NormalTok{predMat }\OtherTok{\textless{}{-}} \FunctionTok{predict}\NormalTok{(knn, testna, }\AttributeTok{type =} \StringTok{"prob"}\NormalTok{)}
\FunctionTok{brier.score}\NormalTok{(obsMat, predMat)}
\end{Highlighting}
\end{Shaded}

\begin{verbatim}
## [1] 1.000356
\end{verbatim}

\hypertarget{nakljuux10dno-gozd}{%
\subsubsection{naključno gozd}\label{nakljuux10dno-gozd}}

\begin{Shaded}
\begin{Highlighting}[]
\FunctionTok{CA}\NormalTok{(observed, predicted)}
\end{Highlighting}
\end{Shaded}

\begin{verbatim}
## [1] 0.5141304
\end{verbatim}

\begin{Shaded}
\begin{Highlighting}[]
\NormalTok{predMat }\OtherTok{\textless{}{-}} \FunctionTok{predict}\NormalTok{(rf, testna, }\AttributeTok{type =} \StringTok{"prob"}\NormalTok{)}
\FunctionTok{brier.score}\NormalTok{(obsMat, predMat)}
\end{Highlighting}
\end{Shaded}

\begin{verbatim}
## [1] 0.7388644
\end{verbatim}

\hypertarget{svm}{%
\subsubsection{SVM}\label{svm}}

\begin{Shaded}
\begin{Highlighting}[]
\FunctionTok{brier.score}\NormalTok{(obsMat, predMat)}
\end{Highlighting}
\end{Shaded}

\begin{verbatim}
## [1] 0.6977981
\end{verbatim}

\begin{Shaded}
\begin{Highlighting}[]
\FunctionTok{CA}\NormalTok{(observed, predicted)}
\end{Highlighting}
\end{Shaded}

\begin{verbatim}
## [1] 0.5141304
\end{verbatim}

\hypertarget{umetne-nevronske-mreux17ee}{%
\subsubsection{Umetne nevronske
mreže}\label{umetne-nevronske-mreux17ee}}

najprej je bilo potrebno normalizirati zvezne atribute v učni in testni
množici nato pa sva dobil te rezultate:

\begin{Shaded}
\begin{Highlighting}[]
\FunctionTok{CA}\NormalTok{(observed, predicted)}
\end{Highlighting}
\end{Shaded}

\begin{verbatim}
## [1] 0.5119147
\end{verbatim}

\begin{Shaded}
\begin{Highlighting}[]
\NormalTok{predMat }\OtherTok{\textless{}{-}} \FunctionTok{predict}\NormalTok{(nn, testna\_scaled, }\AttributeTok{type =} \StringTok{"raw"}\NormalTok{)}
\FunctionTok{brier.score}\NormalTok{(obsMat, predMat)}
\end{Highlighting}
\end{Shaded}

\begin{verbatim}
## [1] 0.8076981
\end{verbatim}

\hypertarget{kobinirani-modeli}{%
\subsubsection{kobinirani modeli}\label{kobinirani-modeli}}

za kombinirane modele sva se odločila uporabiti modele nevrenoske mreže,
naibni bayes in SVM

najprej sva poskusil z glasovanjem in dobila

\begin{Shaded}
\begin{Highlighting}[]
\FunctionTok{CA}\NormalTok{(observed, predicted)}
\end{Highlighting}
\end{Shaded}

\begin{verbatim}
## [1] 0.5484532
\end{verbatim}

nato z uteženim glasovanjem

\begin{Shaded}
\begin{Highlighting}[]
\FunctionTok{CA}\NormalTok{(observed, predicted)}
\end{Highlighting}
\end{Shaded}

\begin{verbatim}
## [1] 0.4796405
\end{verbatim}

nazadnje pa še z boostingom

\begin{Shaded}
\begin{Highlighting}[]
\FunctionTok{CA}\NormalTok{(observed, predicted)}
\end{Highlighting}
\end{Shaded}

\begin{verbatim}
## [1] 0.5048913
\end{verbatim}

na podlagi brier score-a so bili najboljši modeli

\end{document}
